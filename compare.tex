\documentclass[class = professional, twoside, AutoFakeBold=3.17,AutoFakeSlant=0.2]{gdufe_master_thesis}
% class=academic: 学术学位论文
% class=professional: 专业学位论文
% AutoFakeBold=3.17,AutoFakeSlant=0.2  设置字体伪粗体和伪斜体

% \documentclass[forprint,class=academic]{gdufe_master_thesis}% 选项 forprint: 交付打印时添加, 避免彩色链接字迹打印偏淡.

\renewcommand{\thefigure}{\thechapter-\arabic{figure}}
\renewcommand{\thetable}{\thechapter-\arabic{table}}

\usepackage{algorithm}
\usepackage{algpseudocode}

% \setstretch{1}            % 防止 double spacing 影响
% \setlength{\baselineskip}{28pt}

\begin{document}
\setlength{\baselineskip}{20pt}\selectfont

\title{广东财经大学专业学位硕士论文论文题目}

\frontmatter

\chapter*{\ziju{1} 摘要\the\baselineskip}
\addcontentsline{toc}{chapter}{摘要}
\the\baselineskip
行间距:固定值20磅。内容次序为“摘要”二字、摘要内容、关键词。\the\baselineskip

(1)“摘要”二字居中(小二号黑体),两字间空一格(注:“一格”的标准为一个汉字,以下同)。\the\baselineskip

(2)“摘要”二字下空两行,打印摘要内容(小四号宋体),每段开头按照“首行缩进”空两格,标点符号占一格。\the\baselineskip

(3)摘要内容下空一行,开头空两格打印“关键词:”三个字(小四号黑体),其后为关键词内容(小四号宋体),关键词数量为3-8个,每一关键词之间用分号隔开,最后一个关键词后不打标点符号。\the\baselineskip

\chapter*{\bfseries ABSTRACT\the\baselineskip}
\addcontentsline{toc}{chapter}{ABSTRACT}
Time new roman Time new roman Time new roman Time new roman Time new roman Time new roman Time new roman Time new roman Time new roman Time new roman Time new roman Time new roman Time new roman Time new roman Time new roman Time new roman Time new roman\the\baselineskip

Time new roman Time new roman Time new roman Time new roman Time new roman Time new roman Time new roman Time new roman Time new roman Time new roman Time new roman Time new roman Time new roman\the\baselineskip

Time new roman Time new roman Time new roman Time new roman Time new roman Time new roman Time new roman Time new roman\the\baselineskip

Time new roman Time new roman Time new roman Time new roman Time new roman Time new roman Time new roman Time new roman Time new roman Time new roman Time new roman Time new roman Time new roman Time new roman Time new roman Time new roman Time new roman\the\baselineskip

\tableofcontents

\cleardoublepage
\mainmatter

\chapter{章的标题:三号黑体\the\baselineskip}
正文:这篇文章,使用了宋体和 Times New Roman 字体。下面是一些字体命令的效果对比。\the\baselineskip

This article uses SimSun and Times New Roman fonts. Below is a comparison of the effects of some font commands.\the\baselineskip
\section{节的标题:小三号黑体,段前 6 磅,段后 6 磅\the\baselineskip}
正文:这篇文章,使用了宋体和 Times New Roman 字体。下面是一些字体命令的效果对比。\the\baselineskip

This article uses SimSun and Times New Roman fonts. Below is a comparison of the effects of some font commands.\the\baselineskip
\subsection{条的标题:四号黑体,段前 3 磅,段后 3 磅\the\baselineskip}
正文:这篇文章,使用了宋体和 Times New Roman 字体。下面是一些字体命令的效果对比。\the\baselineskip

This article uses SimSun and Times New Roman fonts. Below is a comparison of the effects of some font commands.\the\baselineskip

\subsubsection{款的标题:小四号黑体\the\baselineskip}
正文:这篇文章,使用了宋体和 Times New Roman 字体。下面是一些字体命令的效果对比。\the\baselineskip

\begin{table}[ht]
    \centering
    \begin{tabular}{cccccc}
        \toprule
        测试类型 & {\songti 宋体}             & {\heiti 黑体}                    & {\kaishu 楷书}                    & {\fangsong 仿宋}                    & Times new roman               \\
        \midrule
        伪粗体  & {\bfseries 字体测试}         & {\bfseries\heiti 字体测试}         & {\bfseries\kaishu 字体测试}         & {\bfseries\fangsong 字体测试}         & {\bfseries Font test}         \\
        伪斜体  & {\itshape 字体测试}          & {\itshape\heiti 字体测试}          & {\itshape\kaishu 字体测试}          & {\itshape\fangsong 字体测试}          & {\itshape Font test}          \\
        叠加测试 & {\bfseries\itshape 字体测试} & {\bfseries\itshape\heiti 字体测试} & {\bfseries\itshape\kaishu 字体测试} & {\bfseries\itshape\fangsong 字体测试} & {\bfseries\itshape Font test} \\
        \bottomrule
    \end{tabular}
    \caption{不同字体下伪粗体、伪斜体及叠加效果测试}
\end{table}

\cleardoublepage

本人郑重声明:所呈交的学位论文是本人在导师的指导下独立进行研究工作所取得的成果。除文中已经注明引用的内容外,本论文不含任何其他个人或集体已经发表或撰写过的作品成果。对本文的研究做出重要贡献的个人和集体,均已在文中以明确方式标明。因本学位论文引起的法律后果完全由本人承担。本人郑重声明:所呈交的学位论文是本人在导师的指导下独立进行研究工作所取得的成果。除文中已经注明引用的内容外,本论文不含任何其他个人或集体已经发表或撰写过的作品成果。对本文的研究做出重要贡献的个人和集体,均已在文中以明确方式标明。因本学位论文引起的法律后果完全由本人承担。本人郑重声明:所呈交的学位论文是本人在导师的指导下独立进行研究工作所取得的成果。除文中已经注明引用的内容外,本论文不含任何其他个人或集体已经发表或撰写过的作品成果。对本文的研究做出重要贡献的个人和集体,均已在文中以明确方式标明。因本学位论文引起的法律后果完全由本人承担。本人郑重声明:所呈交的学位论文是本人在导师的指导下独立进行研究工作所取得的成果。除文中已经注明引用的内容外,本论文不含任何其他个人或集体已经发表或撰写过的作品成果。对本文的研究做出重要贡献的个人和集体,均已在文中以明确方式标明。因本学位论文引起的法律后果完全由本人承担。本人郑重声明:所呈交的学位论文是本人在导师的指导下独立进行研究工作所取得的成果。除文中已经注明引用的内容外,本论文不含任何其他个人或集体已经发表或撰写过的作品成果。对本文的研究做出重要贡献的个人和集体,均已在文中以明确方式标明。因本学位论文引起的法律后果完全由本人承担。本人郑重声明:所呈交的学位论文是本人在导师的指导下独立进行研究工作所取得的成果。除文中已经注明引用的内容外,本论文不含任何其他个人或集体已经发表或撰写过的作品成果。对本文的研究做出重要贡献的个人和集体,均已在文中以明确方式标明。因本学位论文引起的法律后果完全由本人承担。本人郑重声明:所呈交的学位论文是本人在导师的指导下独立进行研究工作所取得的成果。除文中已经注明引用的内容外,本论文不含任何其他个人或集体已经发表或撰写过的作品成果。对本文的研究做出重要贡献的个人和集体,均已在文中以明确方式标明。因本学位论文引起的法律后果完全由本人承担。本人郑重声明:所呈交的学位论文是本人在导师的指导下独立进行研究工作所取得的成果。除文中已经注明引用的内容外,本论文不含任何其他个人或集体已经发表或撰写过的作品成果。对本文的研究做出重要贡献的个人和集体,均已在文中以明确方式标明。因本学位论文引起的法律后果完全由本人承担。本人郑重声明:所呈交的学位论文是本人在导师的指导下独立进行研究工作所取得的成果。除文中已经注明引用的内容外,本论文不含任何其他个人或集体已经发表或撰写过的作品成果。对本文的研究做出重要贡献的个人和集体,均已在文中以明确方式标明。因本学位论文引起的法律后果完全由本人承担。本人郑重声明:所呈交的学位论文是本人在导师的指导下独立进行研究工作所取得的成果。除文中已经注明引用的内容外,本论文不含任何其他个人或集体已经发表或撰写过的作品成果。对本文的研究做出重要贡献的个人和集体,均已在文中以明确方式标明。因本学位论文引起的法律后果完全由本人承担。本人郑重声明:所呈交的学位论文是本人在导师的指导下独立进行研究工作所取得的成果。
\the\baselineskip

本人郑重声明:所呈交的学位论文是本人在导师的指导下独立进行研究工作所取得的成果。本人郑重声明:所呈交的学位论文是本人在导师的指导下独立进行研究工作所取得的成果。本人郑重声明:所呈交的学位论文是本人在导师的指导下独立进行研究工作所取得的成果。本人郑重声明:所呈交的学位论文是本人在导师的指导下独立进行研究工作所取得的成果。本人郑重声明:所呈交的学位论文是本人在导师的指导下独立进行研究工作所取得的成果。本人郑重声明:所呈交的学位论文是本人在导师的指导下独立进行研究工作所取得的成果。本人郑重声明:所呈交的学位论文是本人在导师的指导下独立进行研究工作所取得的成果。
    \the\baselineskip

\zihao{-3} 本人郑重声明:所呈交的学位论文是本人在导师的指导下独立进行研究工作所取得的成果。本人郑重声明:所呈交的学位论文是本人在导师的指导下独立进行研究工作所取得的成果。本人郑重声明:所呈交的学位论文是本人在导师的指导下独立进行研究工作所取得的成果。本人郑重声明:所呈交的学位论文是本人在导师的指导下独立进行研究工作所取得的成果。本人郑重声明:所呈交的学位论文是本人在导师的指导下独立进行研究工作所取得的成果。本人郑重声明:所呈交的学位论文是本人在导师的指导下独立进行研究工作所取得的成果。本人郑重声明:所呈交的学位论文是本人在导师的指导下独立进行研究工作所取得的成果。
    \the\baselineskip

{\zihao{-5} 本人郑重声明:所呈交的学位论文是本人在导师的指导下独立进行研究工作所取得的成果。本人郑重声明:所呈交的学位论文是本人在导师的指导下独立进行研究工作所取得的成果。本人郑重声明:所呈交的学位论文是本人在导师的指导下独立进行研究工作所取得的成果。本人郑重声明:所呈交的学位论文是本人在导师的指导下独立进行研究工作所取得的成果。本人郑重声明:所呈交的学位论文是本人在导师的指导下独立进行研究工作所取得的成果。本人郑重声明:所呈交的学位论文是本人在导师的指导下独立进行研究工作所取得的成果。本人郑重声明:所呈交的学位论文是本人在导师的指导下独立进行研究工作所取得的成果。
\the\baselineskip}

本人郑重声明:所呈交的学位论文是本人在导师的指导下独立进行研究工作所取得的成果。本人郑重声明:所呈交的学位论文是本人在导师的指导下独立进行研究工作所取得的成果。本人郑重声明:所呈交的学位论文是本人在导师的指导下独立进行研究工作所取得的成果。本人郑重声明:所呈交的学位论文是本人在导师的指导下独立进行研究工作所取得的成果。本人郑重声明:所呈交的学位论文是本人在导师的指导下独立进行研究工作所取得的成果。本人郑重声明:所呈交的学位论文是本人在导师的指导下独立进行研究工作所取得的成果。本人郑重声明:所呈交的学位论文是本人在导师的指导下独立进行研究工作所取得的成果。
\the\baselineskip

\zihao{-4} 本人郑重声明:所呈交的学位论文是本人在导师的指导下独立进行研究工作所取得的成果。本人郑重声明:所呈交的学位论文是本人在导师的指导下独立进行研究工作所取得的成果。本人郑重声明:所呈交的学位论文是本人在导师的指导下独立进行研究工作所取得的成果。
\the\baselineskip

\textbf{本人郑重声明:所呈交的学位论文是本人在导师的指导下独立进行研究工作所取得的成果。本人郑重声明:所呈交的学位论文是本人在导师的指导下独立进行研究工作所取得的成果。本人郑重声明:所呈交的学位论文是本人在导师的指导下独立进行研究工作所取得的成果。
\the\baselineskip}

{\bfseries 本人郑重声明:所呈交的学位论文是本人在导师的指导下独立进行研究工作所取得的成果。本人郑重声明:所呈交的学位论文是本人在导师的指导下独立进行研究工作所取得的成果。本人郑重声明:所呈交的学位论文是本人在导师的指导下独立进行研究工作所取得的成果。
\the\baselineskip}

本人郑重声明:所呈交的学位论文是本人在导师的指导下独立进行研究工作所取得的成果。本人郑重声明:所呈交的学位论文是本人在导师的指导下独立进行研究工作所取得的成果。本人郑重声明:所呈交的学位论文是本人在导师的指导下独立进行研究工作所取得的成果。
\the\baselineskip

\bfseries 本人郑重声明:所呈交的学位论文是本人在导师的指导下独立进行研究工作所取得的成果。本人郑重声明:所呈交的学位论文是本人在导师的指导下独立进行研究工作所取得的成果。本人郑重声明:所呈交的学位论文是本人在导师的指导下独立进行研究工作所取得的成果。
\the\baselineskip
\end{document}
