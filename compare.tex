\documentclass[class = professional, oneside, AutoFakeBold=3.17,AutoFakeSlant=0.2]{gdufe_master_thesis}
% class=academic: 学术学位论文
% class=professional: 专业学位论文
% AutoFakeBold=3.17,AutoFakeSlant=0.2  设置字体伪粗体和伪斜体

% \documentclass[forprint,class=academic]{gdufe_master_thesis}% 选项 forprint: 交付打印时添加, 避免彩色链接字迹打印偏淡.


\renewcommand{\thefigure}{\thechapter-\arabic{figure}}
\renewcommand{\thetable}{\thechapter-\arabic{table}}

\usepackage{algorithm}
\usepackage{algpseudocode}
\begin{document}
\chapter{章的标题:三号黑体}
正文:这篇文章,使用了宋体和 Times New Roman 字体。下面是一些字体命
令的效果对比。

This article uses SimSun and Times New Roman fonts. Below is a comparison of the
effects of some font commands.
\section{节的标题:小三号黑体,段前 6 磅,段后 6 磅}
正文:这篇文章,使用了宋体和 Times New Roman 字体。下面是一些字体命
令的效果对比。

This article uses SimSun and Times New Roman fonts. Below is a comparison of the
effects of some font commands.
\subsection{条的标题:四号黑体,段前 3 磅,段后 3 磅}
正文:这篇文章,使用了宋体和 Times New Roman 字体。下面是一些字体命
令的效果对比。

This article uses SimSun and Times New Roman fonts. Below is a comparison of the
effects of some font commands.

\subsubsection{款的标题:小四号黑体}
正文:这篇文章,使用了宋体和 Times New Roman 字体。下面是一些字体命
令的效果对比。

This article uses SimSun and Times New Roman fonts. Below is a comparison of the
effects of some font commands.

\begin{table}[ht]
    \centering
    \begin{tabular}{cccccc}
        \toprule
        测试类型 & {\songti 宋体}             & {\heiti 黑体}                    & {\kaishu 楷书}                    & {\fangsong 仿宋}                    &Times new roman\\
        \midrule
        伪粗体  & {\bfseries 字体测试}          & {\bfseries\heiti 字体测试}          & {\bfseries\kaishu 字体测试}          & {\bfseries\fangsong 字体测试}          &{\bfseries Font test}\\
        伪斜体  & {\itshape 字体测试}           & {\itshape\heiti 字体测试}           & {\itshape\kaishu 字体测试}           & {\itshape\fangsong 字体测试}           &{\itshape Font test}\\
        叠加测试 & {\bfseries\itshape 字体测试} & {\bfseries\itshape\heiti 字体测试} & {\bfseries\itshape\kaishu 字体测试} & {\bfseries\itshape\fangsong 字体测试} &{\bfseries\itshape Font test} \\
        \bottomrule
    \end{tabular}
    \caption{不同字体下伪粗体、伪斜体及叠加效果测试}
\end{table}


\end{document}