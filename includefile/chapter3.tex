\chapter{测试\the\baselineskip}

此章节基于武汉大学学位论文模版测试部分修改. \the\baselineskip

This chapter is modified based on the test section of the Wuhan University thesis template.

\section{文字测试\the\baselineskip}


中文宋体正常 \textbf{中文宋体加粗} \textit{中文宋体斜体}\the\baselineskip

English \textbf{English} \textit{English}\the\baselineskip

\section{公式\the\baselineskip}

\subsection{算符、希腊字母\the\baselineskip}
\begin{gather*}
    \frac{abc}{xyz} \frac{\d^n y}{\d x^n} \int_{a}^{b}  \,\d x \lim_{x \to \infty}  \sum_{n = 1}^{\infty}  \iint \oint \prod \coprod  \\
    \alpha\beta\gamma\delta \epsilon \varepsilon \zeta \eta \theta \vartheta \iota \kappa \lambda \mu \nu \xi o \pi \varpi \rho \varrho \sigma \varsigma \tau \upsilon \phi \varphi \chi \psi \omega\\
    A B \Gamma \varGamma \Delta \varDelta E Z H \Theta \varTheta I K \Lambda \varLambda M N \Xi \varXi O \Pi \varPi P \Sigma \varSigma T \Upsilon \varUpsilon \Phi \varPhi X \Psi \varPsi \Omega \varOmega
\end{gather*}

\subsection{几类数学字母表\the\baselineskip}

\begin{itemize}
    \item \verb|\mathcal|: $\mathcal{ABCDEFGHIJKLMNOPQRSTUVWXYZ}$ \the\baselineskip
    \item \verb|\mathscr|: $\mathscr{ABCDEFGHIJKLMNOPQRSTUVWXYZ}$ \the\baselineskip
    \item \verb|\mathbb|: $\mathbb{ABCDEFGHIJKLMNOPQRSTUVWXYZ}$ \the\baselineskip
\end{itemize}


\subsection{(不)带编号单行公式 \the\baselineskip}

Use \texttt{equation} environment: \the\baselineskip
\begin{equation}
    a^2 + b^2 = c^2. \the\baselineskip
\end{equation}
\the\baselineskip

Use \texttt{equation*} environment or \texttt{\textbackslash[...\textbackslash]}: \the\baselineskip
\[ a^2 + b^2 = c^2. \the\baselineskip\]
\the\baselineskip
\subsection{(不)带编号多行公式}

Use \texttt{align} environment:
\begin{align}
    S_n & = 1 + 2 + \cdots + n \\
        & = \frac12 n(n+1).
\end{align}

Use \texttt{align*} environment:
\begin{align*}
    T_n & = 1^3 + 2^3 + \cdots + n^3         \\
        & = \biggl(\frac{n(n+1)}{2}\biggr)^2 \\
        & = S_n^2.
\end{align*}

\subsection{矩阵}

\[
    \begin{pmatrix}
        a_{11} & a_{22} & a_{33} \\
        a_{21} & a_{22} & a_{23} \\
        a_{31} & a_{32} & a_{33} \\
    \end{pmatrix} \quad
    \begin{vmatrix}
        a_{11} & a_{22} & a_{33} \\
        a_{21} & a_{22} & a_{23} \\
        a_{31} & a_{32} & a_{33} \\
    \end{vmatrix} \quad
    \begin{bmatrix}
        a_{11} & a_{22} & a_{33} \\
        a_{21} & a_{22} & a_{23} \\
        a_{31} & a_{32} & a_{33} \\
    \end{bmatrix} \quad
    \begin{Bmatrix}
        a_{11} & a_{22} & a_{33} \\
        a_{21} & a_{22} & a_{23} \\
        a_{31} & a_{32} & a_{33} \\
    \end{Bmatrix}\]

\section{脚注测试}

测试 \footnote{眼看他起朱楼,眼看他宴宾客,眼看他楼塌了。这青苔碧瓦堆,俺曾睡风流觉,将五十年兴亡看饱。
    金粉未消亡,闻得六朝香,满天涯烟草断人肠。怕催花信紧,风风雨雨,误了春光。}

测试 \footnote[3]{君不见,左纳言,右纳史,朝承恩,暮赐死。行路难,不在水,不在山,只在人情反覆间!}

\section{引用测试}

\subsection{参考文献}

测试 \cite{huangzh,anon-cn1,anon-cn2,anon-cn3,anon-cn4,anon-cn5}

测试 \cite*{anon-en1,anon-en2,anon-en3,anon-en4,anon-en5}

注意手写参考文献和使用 bib 文件的样式有顶格不顶格的区别,由于没有详细要求,选一使用即可。

\section{图表测试}

\begin{figure}[ht]
    \centering
    \includegraphics[width = 5cm]{gdufelogo.png}
    \caption{广东财经大学校徽}
    \label{fig:广东财经大学校徽}
\end{figure}

引用图~\ref{fig:广东财经大学校徽}

多图测试1
\begin{figure}[H]
    \centering
    \includegraphics[width=0.3\textwidth]{lake.jpeg}\hfill
    \includegraphics[width=0.3\textwidth]{lake.jpeg}\hfill
    \includegraphics[width=0.3\textwidth]{lake.jpeg}
    \caption{多图片1:珠江三角洲投资管理体制市场化改革(1985---2005)\the\baselineskip}
    {\zihao{5} 资料来源:xxx\the\baselineskip}
    \label{fig:MultiPic1}
\end{figure}

多图测试2
\begin{figure}[h]
    \centering
    \includegraphics[width=0.5\textwidth]{lake.jpeg}\\
    \includegraphics[width=0.5\textwidth]{lake.jpeg}\\
    \includegraphics[width=0.5\textwidth]{lake.jpeg}
    \caption{多图片2}
    \label{fig:MultiPic2}
\end{figure}

子图测试
\begin{figure}[htbp]
    \centering

    \begin{subcaptionbox}{子图A标题\label{fig:subfig_a}\the\baselineskip}[0.45\textwidth]
        {\includegraphics[width=\linewidth]{lake.jpeg}}
    \end{subcaptionbox}
    \hfill
    \begin{subcaptionbox}{子图B标题\label{fig:subfig_b}}[0.45\textwidth]
        {\includegraphics[width=\linewidth]{lake.jpeg}}
    \end{subcaptionbox}

    \caption{整体图标题\the\baselineskip}
    \label{fig:subfigtest}
\end{figure}

如图~\ref{fig:subfigtest} 所示,其中~\subref{fig:subfig_a} 是子图A, \subref{fig:subfig_b} 是子图B.
注意使用 \verb|\ref{}| \ref{fig:subfig_a} 和 \verb|\subref{}| \subref{fig:subfig_a}命令来引用子图的区别.


表格测试
\begin{table}[ht]
    \centering
    \caption{%
        简单的表格和引用 abc 123 %\cite{whu-bachelor:1}
    }
    \label{table:简单的表格}
    \begin{tabular}{cc}
        \hline
        a  & b  \\ \hline
        c  & d  \\ \hline
        测试 & 文本 \\ \hline
    \end{tabular}
\end{table}

\begin{table}[ht]
    \centering
    \caption{%
        三线表格和引用 abc 123 %\cite{whu-bachelor:1}
    }
    \label{table:三线表格}
    \begin{tabular}{cc}
        \toprule
        列名1 & 列名2 \\ \midrule
        a   & b   \\
        c   & d   \\
        测试  & 文本  \\ \bottomrule
    \end{tabular}
\end{table}

引用表~\ref{table:简单的表格}
引用表~\ref{table:三线表格}



\section{已定义好的一些数学定理环境}

\begin{defi}[测度]
    (参见文献xxx) 这是一段文字 $E = m c^2$  (中文括号)和 (西文括号)
\end{defi}

\begin{theo}
    这是一段文字 $E = m c^2$
\end{theo}

\begin{proof}
    这是一段文字 $E = m c^2$
\end{proof}

\begin{proof}[定理xx的证明]
    这是一段文字 $E = m c^2$
\end{proof}

\begin{exam}
    这是一段文字 $E = m c^2$
\end{exam}

\begin{prop}
    这是一段文字 $E = m c^2$
\end{prop}

\begin{coro}
    这是一段文字 $E = m c^2$
\end{coro}

\begin{lemm}
    这是一段文字 $E = m c^2$
\end{lemm}

\begin{rema}
    这是一段文字 $E = m c^2$
\end{rema}

\begin{theo}[Banach-Steinhaus]\label{thm:test}
    设 $E$ 是 Banach 空间, $F$ 是赋范空间, $(u_i)_{i\in I}$ 是一族从 $E$ 到 $F$ 的有界线性算子,
    即 $(u_i)_{i\in I}\subset \mathcal{B}(E,F)$. 若对每一点 $x\in E$, 有
    $\sup_{i\in I} \|u_i(x)\|<\infty$, 则
    \begin{equation}\label{eq:test1}
        \sup_{i\in I} \|u_i\| < \infty.
    \end{equation}
\end{theo}

我想引用定理~\ref{thm:test} 和公式~\ref{eq:test1}

定理括号测试:

\begin{theo}
    测试
    \begin{enumerate}
        \item 中文(括号)没输入空格的效果
        \item 中文 (括号) 输入空格的效果
        \item 西文(括号)没输入空格的效果
        \item 西文 (括号) 输入空格的效果
    \end{enumerate}
\end{theo}

\begin{proof}
    test
    \[
        \int_{0}^{1} x^2 \d x
    \]
\end{proof}

% \begin{proof}
%   test
%   \[
%     \int_{0}^{1} x^2 \d x \qedhere
%   \]
% \end{proof}

\section{字体测试}
字体测试:

\begin{table}[ht]\centering
    \begin{tabular}{lll}
        \toprule
        命令               & 宋体效果                & Times New Roman 效果                       \\
        \midrule
        \verb|\rmfamily| & {\rmfamily 宋体罗马字族}  & {\rmfamily Times New Roman Roman Family} \\
        \verb|\sffamily| & {\sffamily 宋体无衬线字族} & {\sffamily Times New Roman Sans Serif}   \\
        \verb|\ttfamily| & {\ttfamily 宋体打字机}   & {\ttfamily Times New Roman Typewriter}   \\
        \verb|\bfseries| & {\bfseries 宋体加粗}    & {\bfseries Times New Roman Bold}         \\
        \verb|\itshape|  & {\itshape 宋体意大利}    & {\itshape Times New Roman Italic}        \\
        \verb|\slshape|  & {\slshape 宋体倾斜}     & {\slshape Times New Roman Slanted}       \\
        \bottomrule
    \end{tabular}
    \caption{宋体与 Times New Roman 字体命令效果对比}
\end{table}

\begin{table}[ht]
    \centering
    \begin{tabular}{cccccc}
        \toprule
        测试类型 & {\songti 宋体}             & {\heiti 黑体}                    & {\kaishu 楷书}                    & {\fangsong 仿宋}                    & Times new roman               \\
        \midrule
        伪粗体  & {\bfseries 字体测试}         & {\bfseries\heiti 字体测试}         & {\bfseries\kaishu 字体测试}         & {\bfseries\fangsong 字体测试}         & {\bfseries Font test}         \\
        伪斜体  & {\itshape 字体测试}          & {\itshape\heiti 字体测试}          & {\itshape\kaishu 字体测试}          & {\itshape\fangsong 字体测试}          & {\itshape Font test}          \\
        叠加测试 & {\bfseries\itshape 字体测试} & {\bfseries\itshape\heiti 字体测试} & {\bfseries\itshape\kaishu 字体测试} & {\bfseries\itshape\fangsong 字体测试} & {\bfseries\itshape Font test} \\
        \bottomrule
    \end{tabular}
    \caption{不同字体下伪粗体、伪斜体及叠加效果测试}
\end{table}

Lorem ipsum dolor sit amet, consectetur adipiscing elit, sed do eiusmod tempor incididunt ut labore et dolore magna aliqua. Ut enim ad minim veniam, quis nostrud exercitation ullamco laboris nisi ut aliquip ex ea commodo consequat. Duis aute irure dolor in reprehenderit in voluptate velit esse cillum dolore eu fugiat nulla pariatur.Lorem ipsum dolor sit amet, consectetur adipiscing elit, sed do eiusmod tempor incididunt ut labore et dolore magna aliqua. Ut enim ad minim veniam, quis nostrud exercitation ullamco laboris nisi ut aliquip ex ea commodo consequat. Duis aute irure dolor in reprehenderit in voluptate velit esse cillum dolore eu fugiat nulla pariatur.

\section{算法}

\begin{algorithm}
    \caption{An algorithm with caption}\label{alg:cap}
    \begin{algorithmic}
        \Require $n \geq 0$
        \Ensure $y = x^n$
        \State $y \gets 1$
        \State $X \gets x$
        \State $N \gets n$
        \While{$N \neq 0$}
        \If{$N$ is even}
        \State $X \gets X \times X$
        \State $N \gets \frac{N}{2}$  \Comment{This is a comment}
        \ElsIf{$N$ is odd}
        \State $y \gets y \times X$
        \State $N \gets N - 1$
        \EndIf
        \EndWhile
    \end{algorithmic}
\end{algorithm}

引用 \autoref{alg:cap}