\chapter{撰写、编译、打印}

\section{具体使用步骤}

\begin{description}

    \item[Step 1]  进入 includefile 文件夹,  打开 abstract.tex, acknowledgment.tex 这两个文档, 分别填写 (1) 中文摘要、英文摘要, (2) 致谢.

    \item[Step 2]  打开主文档 gdufe\_master\_thesis\_template.tex, 填写题目、作者等等信息, 撰写正文. 主文档的文件名可以修改,但要求用英文名.

    \item[Step 3]  使用 XeLaTeX 编译. 具体见 \ref{sec-compile} 节.

\end{description}

\section{编译的方法}\label{sec-compile}

默认使用 XeLaTeX 编译, 直接生成~pdf 文件.

若另存为新文档, 请确保文档保存类型为 \verb|:UTF-8|. 当然目前很多编辑器默认文字编码为 UTF-8.
WinEdt 9.0 之后的版本都是默认保存为 UTF-8 的.

使用~XeLaTeX 编译, 直接生成~pdf 文件.
pdf 文件也可以反向搜索! 双击~pdf 中要修改的文字, 将直接跳转到源文件中相应位置.

如果使用 bib 文件管理参考文献,需要经过
\begin{enumerate}
    \item XeLaTeX

    \item Biber

    \item XeLaTeX

    \item XeLaTeX
\end{enumerate}
四次编译

% \section{文档类型选择}
% 文档类型有 2 种情形:

% \begin{table}[ht]\centering
%     \begin{tabular}{ll}
%         \hline
%         \verb|\documentclass{gdufe\_term\_thesis}|           & 学年论文电子版 \\
%         \verb|\documentclass[forprint]{gdufe\_term\_thesis}| & 毕业论文打印版 \\
%         \hline
%     \end{tabular}
% \end{table}
% 相关解释见下节.

% 2025.5.15更新:只有一个类型 gdufe\_master\_thesis.

\section{打印的问题}
\begin{enumerate}[i)]
    %  \item  论文要求\colorbox{yellow}{单面打印}.
    \item  关于文档选项 forprint: 交付打印时, 建议加上选项 forprint, 以消除链接文字之彩色, 避免打印字迹偏淡.
    \item  打印时留意不要缩小页面或居中. 即页面放缩方式应该是``无''(Adobe Reader XI 是选择``实际大小'').
          有可能页面放缩方式默认为``适合可打印区域'', 会导致打印为原页面大小的 $97\%$.
          文字不要居中打印, 是因为考虑到装订, 左侧的空白留得稍多一点(模板已作预留).
    \item  遗留问题: 封面需要打印部重新制作.  校内打印部通常有现成的模板.
          我们自己做的封面, 打印部不一定好用.
\end{enumerate}
%如果不是彩色打印机, 请在打印时, 选择将彩色打印为黑白, 否则彩色文字打出的墨迹会偏淡.

\textbf{问}: {\kaishu 生成 PDF 文件时,不能去掉目录和文章的引用彩色方框,请问怎么解决?}

\textbf{答}: {\kaishu 方框表示超级链接, 只在电脑上看得见. 实际打印时, 是没有的. 另外, 文档类型加选项 forprint 之后, 这些框框会隐掉的. }

% \vfill
\section{文档中的若干命令}
\begin{enumerate}
    \item 若删去参考文献后的\verb|\thispagestyle{plain}|命令则会导致参考文献页显示页眉
\end{enumerate}
